\documentclass[11pt, oneside]{article}   	% use "amsart" instead of "article" for AMSLaTeX format
\usepackage{geometry}                		% See geometry.pdf to learn the layout options. There are lots.
\geometry{letterpaper}                   		% ... or a4paper or a5paper or ... 
%\geometry{landscape}                		% Activate for for rotated page geometry
%\usepackage[parfill]{parskip}    		% Activate to begin paragraphs with an empty line rather than an indent
\usepackage{graphicx}				% Use pdf, png, jpg, or eps� with pdflatex; use eps in DVI mode
								% TeX will automatically convert eps --> pdf in pdflatex		
\usepackage{amssymb}
\usepackage{setspace}
\onehalfspace


\title{Second Colloquium Writeup}
\author{Bill DeRose}
%\date{}							% Activate to display a given date or no date

\begin{document}
\maketitle
This past Wednesday Professor David Bachman from Pitzer College discussed  the merits and shortcomings of four types of 3D printing processes before giving a tutorial on how to construct a simple 3D model torus using the Rhinoceros software. The four different 3D printing methodologies Bachman discussed were selective laser sintering (SLS), stereolithography (SLA), paper,  and fused deposition modeling (FDM). SLS uses a laser to heat up layers of powder into a solid. The model is essentially traced out by the laser's heat which melds the powder together. Excess powder is removed at the end to reveal the model inside. SLA is an additive manufacturing process by which thin layers of material are melded together on top of one another. This method, too, requires a support structure around the final model. To my surprise, paper is also suitable as a medium for 3D printing. The colors possible on the 3D paper models are unsurpassed by any of the other printing methods. Finally, Bachman discussed the merits of FDM 3D printers which are more accessible to hobbyists because of their lower price points. Like SLA, FDM melts cylinders of plastic and lays them down in layers to form the model. One of the applications of 3D printing Bachman highlighted was the recently installed 3D printer on the space station which allows parts to be manufactured on-site rather than shipped from Earth. Bachman  briefly  touched on the moral issues 3D printing presents, such as the power it places in civilian hands should they choose to 3D print guns or other weapons. Finally, Bachman closed with a tutorial of his preferred modeling software showing the audience how to parameterize a torus. Overall, I thought the presentation was interesting though I wish Professor Bachman had shown us a few more powerful tools in the 3D modeling library. 
\end{document}  