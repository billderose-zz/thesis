\documentclass[11pt, oneside]{article}   	% use "amsart" instead of "article" for AMSLaTeX format
\usepackage{geometry}                		% See geometry.pdf to learn the layout options. There are lots.
\geometry{letterpaper}                   		% ... or a4paper or a5paper or ... 
%\geometry{landscape}                		% Activate for for rotated page geometry
%\usepackage[parfill]{parskip}    		% Activate to begin paragraphs with an empty line rather than an indent
\usepackage{graphicx}				% Use pdf, png, jpg, or eps� with pdflatex; use eps in DVI mode
								% TeX will automatically convert eps --> pdf in pdflatex		
\usepackage{amssymb}
\usepackage{setspace}
\onehalfspace


\title{First Colloquium Writeup}
\author{Bill DeRose}
%\date{}							% Activate to display a given date or no date

\begin{document}
\maketitle
Professor Shahriari presented on intersecting families of sets, permutations, and vector spaces in his colloquium talk on September 10th. He began with a brief overview of �stars� in the context of sets before moving on to more interesting results in the field, including one of his own. We say a collection of subsets is a star if a fixed element is contained in each subset. I found the initial definitions very tractable and appreciated the questions Professor Shahriari posed (e.g. ``What is the size of the biggest star in the power set?") to motivate the rest of his talk. Starting small and working towards more complicated concepts is invaluable; rather than catering to the most enlightened audience members and diving into the most esoteric parts of the topic, Professor Shahriari gave everyone an opportunity to grasp the basics before introducing more complex concepts. As a result, I found the first half of the presentation  very approachable and was able to follow most of the lemmas and theorems contained therein. Professor Shahriari's presentation style was conducive to my learning style: he talked slowly enough so the listener could absorb what was being said. Although I understood the material covered earlier in the talk better, I think the most interesting facet of the entire presentation was how we can define intersecting families on vector spaces. Ultimately, I was not able to follow along for the entire presentation and lost track of the math about three-quarters of the way through. Nevertheless, I found the topic interesting and the method of presentation worthy of emulation.  



\end{document}  